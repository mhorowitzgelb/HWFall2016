\documentclass[]{article}

\usepackage{amssymb}
\usepackage{graphicx}

\newcommand*{\QEDA}{\hfill\ensuremath{\blacksquare}}%


\title{\vspace{-2cm} Assignment 3}
\author{Max Horowitz-Gelb, Benjamin Chylla}

\begin{document}
\maketitle

\section*{1}
Solution for problem\\
Corrected augmented path\\
rp pq qd ds width: 2\\
Corrected augmented path\\
rq qd ds width: 4\\
Corrected augmented path\\
ra ad dc cs width: 1\\
Corrected augmented path\\
ra ac cs width: 3\\
Corrected augmented path\\
ra ac cb bs width: 2\\
Corrected augmented path\\
rp pb bs width: 3\\
Corrected augmented path\\
ra ac cq qb bs width: 1\\
Final flow:\\
pq, u: 2, x: 2\\
pb, u: 3, x: 3\\
qp, u: 1, x: 0\\
qb, u: 2, x: 1\\
qd, u: 6, x: 6\\
ra, u: 9, x: 7\\
rp, u: 6, x: 5\\
rq, u: 4, x: 4\\
ds, u: 6, x: 6\\
dc, u: 1, x: 1\\
ad, u: 1, x: 1\\
ac, u: 8, x: 6\\
cs, u: 4, x: 4\\
cq, u: 1, x: 1\\
cb, u: 2, x: 2\\
ba, u: 1, x: 0\\
bs, u: 8, x: 6\\

\includegraphics[scale=0.5]{Exercise1}

\section*{3}
\subsection*{Corollary 3.8}	
If $x$ is a feasible $(r$-$s)$-flow and $\delta(R)$ is an $(r$-$s)$-cut, then $x$ is maximum and $\delta(R)$ is minimum if and only if
$x_e = u_e$ for all $e \in \delta(R)$ and $x_e = 0$ for all $e \in \delta(\bar{R})$.
\subsection*{Proof}
If $\delta(R)$ is a minimum $(r$-$s)$-cut then by the minimum cut maximum flow theorem, the maximum flow is equal to $u(\delta(R))$.  
Then by corollary 3.3 and the fact that our flow is maximum, $x(\delta(R)) - x(\delta(\bar{R})) = f_x(s) = u(\delta(R))$. And combining with the fact that $x_e$ must be non-negative, and $x_e \leq u_e$, then it must be true that $x_e = u_e$ for all $e \in R$ and $x_e = 0$ for all $e \in \bar{R}$ for the above equation to be correct.

Vice-versa if $x$ is a feasible flow and for some $(r$-$s)$ cut $\delta(R)$ we have that $x_e = u_e$ for all $e \in R$ and $x_e = 0$ for all $e \in \bar{R}$ then $\delta(R)$ must be a minimum $(r$-$s)$ cut. If not then let $\delta(R')$ be some minimum cut. Then by corollary 3.3, $x(\delta(R)) - x(\delta(\bar{R}) = x(\delta(R)) = u(\delta(R)) = f_x(s) > u(\delta(R')$ which creates a contradiction since our flow is greater than our minimum cut. And since $f_x(s) = u(\delta(R)$, then by the mininum cut maximum flow theorem our flow is maximum.
\QEDA

\section*{4}
\includegraphics[scale=0.5]{exercise4}

As one can see the figure above shows one maximum flow for this graph. As well, there are the labeled set of nodes $R = \{r,q,a\}$. Via visual inspection of our flow $x$, $R$ is the only set of nodes such that $\delta(R)$ is an $(r$-$s)$-cut, $x_e = u_e$ for all $e \in \delta(R)$ and $x_e = 0 $ for all $e \in \delta(\bar{R})$. Since by corrollary 3.8 a cut must have this property to be a minimum $(r$-$s)$-cut for any maximum flow, including $x$, then $\delta(R)$ is only minimum $(r$-$s)$ cut.  

\end{document}
