\documentclass[]{article}

\usepackage{amssymb}

\newcommand*{\QEDA}{\hfill\ensuremath{\blacksquare}}%

\title{Assignment 3}
\author{Max Horowitz-Gelb, Benjamin Chylla}

\begin{document}
\maketitle

\section*{1}
\section*{3}
\subsection*{Corollary 3.8}	
If $x$ is a feasible $(r$-$s)$-flow and $\delta(R)$ is an $(r$-$s)$-cut, then $x$ is maximum and $\delta(R)$ is minimum if and only if
$x_e = u_e$, for all $e \in \delta(R)$ and $x_e = 0$ for all $e \in \delta(\bar{R})$.
\subsection*{Proof}
If $\delta(R)$ is a minimum $(r$-$s)$-cut then by the minimum cut maximum flow theorem, the maximum flow is equal to $u(\delta(R))$.  
Then by corollary 3.3 and the fact that our flow is maximum $x(\delta(R)) - x(\delta(\bar{R})) = f_x(s) = u(\delta(R))$. And combining with the fact that $x_e$ must be non-negative, and $x_e \leq u_e$, then it must be true that $x_e = u_e$ for all $e \in R$ and $x_e = 0$ for all $e \in \bar{R}$.

Vice-versa if $x$ is a maximum feasible flow and for some $(r$-$s)$ cut $\delta(R)$ we have that $x_e = u_e$ for all $e \in R$ and $x_e = 0$ for all $e \in \bar{R}$ then $\delta(R)$ must be a minimum $(r$-$s)$ cut. If not then let $\delta(R')$ be some minimum cut. Then by corollary 3.3, $x(\delta(R)) - x(\delta(\bar{R}) = x(\delta(R)) = u(\delta(R)) = f_x(s) > u(\delta(R')$ which creates a contradiction since our flow is greater than our minimum cut.
\QEDA

\section*{4}


\end{document}
