\documentclass{article}
\title{Math 425 Assignment 2}
\author{Max Horowitz-Gelb, Tushar Verma}
\date{October 30, 2016}
\usepackage{graphicx}
\usepackage{amsmath}
\DeclareMathOperator*{\argmin}{argmin}

\setlength{\parindent}{0pt}
\setlength{\parskip}{10pt}

\begin{document}
\maketitle
\section*{Problem 1}
\includegraphics[scale=0.5]{tree}

In this figure the minimum spanning tree is labeled with blue lines and the tree created of least cost dipaths is labeled with red lines. These two trees are clearly different from eachother. As one can see the MST does not contain all least cost dipaths since the cost from $r$ to the bottom right node is of cost 3 and there is a cheaper path of cost 2. Conversely the cost of the tree made from the minimum cost dipaths is of cost 5 but the MST is of cost 4, therefore the minimum dipath tree is not an MST.
\section*{Problem 2}
Given a digraph $G = (V,E)$, costs $c_e$ for every $e \in E$ and disjoint sets $R,S \subset V$. Then the problem of finding a least cost dipath starting from any node in $R$ and ending in any node in $S$ can be reduced to solving one ordinary shortest path problem by lightly modifying $G$. Simply make a new graph by adding a special node $\gamma$ to $V$ and edges from $\gamma$ to all nodes in $R$, all with cost $0$. Call this new graph  $G'$.

If G has no negative cost dicircuits, then if we run ordinary least cost dipaths on $G'$ with $r = \gamma$ and achieve a feasible potential $y$ and least cost paths
described by 
$p$, our least cost dipath from $R$ to $S$ in $G$ is a dipath that ends in $t = \argmin\limits_{v \in S} y_v$. Its cost is $y_t$ and its path is the path in $p$ ending with $t$ with node $\gamma$ and outgoing edge from $\gamma$ removed.

First note that $G$ has no negative cost dicircuits if and only if $G'$ has none since $\gamma$ has only outgoing edges and therefore cannot be part of a circuit.

We shall prove this is a valid reduction via contradiction. Assume that there exists a path, $x$, from node $a \in R$ to $b \in S$ in $G$ with less cost than any path found from ordinary least cost dipaths on $G'$ with $r=\gamma$. Since there there is a 0 cost edge from $\gamma$ to $a$ then $y_a \leq 0$. Then if the cost of $x$ is less than all paths found in $G'$ then the cost of $x$ is less than $y_b$. But since $x$ exists in $G'$, we could achieve a path from $\gamma$ to $b$ of cost $y_a + cost(x)$ which is less than $y_b$. This creates a contradiction. Therefore the minimum cost dipath from $\gamma$ to $S$ has a cost less than or equal to the least cost dipath from $R$ to $S$ in $G$. 

And since $\gamma$ has only outgoing edges of cost $0$ then the least cost dipath from $\gamma$ to $S$ must be the same cost as the least cost dipath from $R$ to $S$ in $G$. So by removing $\gamma$ from this path, we obtain a least cost dipath from $R$ to $S$ in $G$.



Finally if $G$ had a negative cost dicircuit then so would $G'$ and running ordinary least cost dipaths on $G'$ would discover this. This would imply there is no solution to the problem.
 
\section*{Problem 3}
Suppose that we are give a shortest path problem on a digraph $G$ such that a node $w\neq r$ is incident with two arcs $a$ and $b$. Then we can answer the problem with a smaller digraph in three different ways. 

First note that without loss of generality we may assume there exists a dipath from $r$ to $v$ for all $v \in V$, since if this were not the case for some $v$ then we could add an arc $rv$ with cost $\sum_E |c_e|$. This arc would never be used in any least cost dipath from $r$ to $x \neq v$.

\subsection*{Case 1}
If $w$ has only outgoing arcs, then clearly there is no path from $r$ to $w$. 
Hence we can solve the shortest dipath problem on $G$ with $w$ and its incident arcs removed and then simply set $y_w = \inf$ and $p(w) = -1$. 

\subsection*{Case2}
If $w$ has only incoming arcs, then clearly there is no path from $r$ to $v$ where $v \neq w$ and $w$ is in said path. Therefore we can do ordinary least cost dipaths on a graph with $w$ and its incident arcs removed to get a $y$ and $p$ and $p$ will contain the least cost dipath from $r$ to $v$ for all $v \in V \setminus \{w\}$.

Then let $a = \alpha w$ and $b = \beta w$. Then clearly since these are the only two incoming arcs for $w$ then the last arc in a path from $r$ to $w$ must be one of these two arcs. And to minimize the cost of a path from $r$ to $w$ it must then contain the minimum path from $r$ to $\beta$ or $r$ to $\alpha$. So we set $y_w = min \{ y_\beta + c_b, y_\alpha + c_a\} $ and
$p(w) = v$ such that $(v,e) \in \{(\alpha,a), (\beta,b)\}$ and $y_w = y_v + c_e$.

\subsection*{Case3}
If $w$ has one incoming arc and one outgoing arc, then we can remove parts of the graph to solve a smaller problem. Let $a = \alpha w$ and let $b = w \beta$ and let $f$ be an arc from $\alpha$ to $\beta$. 



Note without loss of generality we may assume $f$ exists, since if it did not we could add an arc $\alpha \beta$ with cost equal to $\sum_{E} |c_e|$. This arc will then never be in least cost dipath from $r$ to any other node in $V$, so it wouldn't affect our problem.

Then there are 3 cases.

\subsubsection*{Case 3a}
$f$ does not exist. 

In this case we can add an arc $f'=\alpha \beta$ of cost $c_a + c_b$ and remove $w$ and its incident arcs from the graph. If there was any least cost dipath from $r$ to any node other than $w$ that contained $w$ in in it, then the part of the path containing $w$ would have to be of the form $a, w, b$. Then replacing $a, w, b$ with $f'$ would result in another dipath of same cost. Then we could run ordinary least cost dipaths on our transformed graph to get $y$ and $p$. Since as we've shown, our transformation could not affect the cost of any least cost dipath, then we may simply check if $p(\beta) = \alpha$ and if so set $p(\beta) = w$. Then set $p(w) = \alpha$ and $y_w = y_\alpha + c_a$. This would give us our solution.

\subsubsection*{Case 3a}
$c_f > c_a + c_b$. 

In this case any least cost dipath would never include $f$ since we could simply replace $f$ with $a, w, b$ and
achieve a lower cost. Therefore we can remove $f$ and run ordinary least cost dipaths on the subgraph to get the final solution.



\subsubsection*{Case 3b}
$c_f \leq c_a + c+b$

By similar logic to case 1 here we would rather use $f$ than go through $w$. So we simply remove $w$ and run ordinary least cost dipaths on the resultant graph to aquire $y$ and $p$. Then if $y_\alpha \neq \inf$ then set $y_w = y_a + c_a$ and $p(w) = \alpha$. 

This similarly accounts for dicircuits. 

In both cases there are a constant number of operations so the reduction costs are $O(1)$. Therefore this whole procedure is the same time complexity as running ordinary least cost dipaths on a graph with $n-1$ nodes and $m-2$ edges. 


\end{document}

