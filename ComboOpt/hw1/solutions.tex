\documentclass{article}
\usepackage{amssymb}
\newcommand*{\QEDA}{\hfill\ensuremath{\blacksquare}}%
\title{Math 425 Assignment 1}
\author{Max Horowitz-Gelb}
\date{}
\setlength{\parskip}{\baselineskip}%
\setlength{\parindent}{0pt}
\begin{document}
\maketitle
\section{}
\section{}
\textbf{Lemma A:}

\textbf{Statement:} A connected subgraph of $n$ nodes requires at least $n-1$ edges.

\textbf{Proof:} Consider a graph $G = (V,E)$. 

Let $G_n$ be the set of all connected subgraphs of $G$ with $n$ nodes. And let $G'_n$ be a subgraph of $G_n$ such that the number of edges in $G'_n$ is minimized. Then let $E'_n = E(G'_n)$ and $V'_n = V(G'_n)$.

First we shall consider the base case of $n = 1$ nodes. Here it is trivial to see by definition of a graph that $E'_1 = 0$. 

Now assume that $E'_b = \alpha$. Then it must be true that $E'_{b+1} \geq \alpha + 1$ since for any subgraph in $G_b$, by definition of a graph, requires at least one new edge to connect a new node.

Therefore $E'_n \geq n+1$, hence proving our statement. 
\QEDA

\textbf{Lemma 2.2}

\textbf{Statement:} Let $G$ be a connected graph with $n$ nodes. Then $G$ is a spanning tree if and only if it has exactly $n-1$ nodes.

\textbf{Proof:} Let $S_n$ be the set of all spanning trees of all subgraphs of $G$ containing $n$ nodes where a spanning tree is possible. Then let $|E_n|$ be the number of edges of used for each subgraph in $S_n$ assuming they are all the same. 

Clearly $|E_1| = 0$ by definition of a graph.

Then if $S_n = n-1$ then $S_{n+1} = n$. 
This is because Lemma A implies $S_{n+1] \geq n$.     


 
\end{document}