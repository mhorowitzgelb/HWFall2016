\documentclass{article}
\usepackage{amssymb}
\newcommand*{\QEDA}{\hfill\ensuremath{\blacksquare}}%
\title{Math 425 Assignment 1}
\author{Max Horowitz-Gelb}
\date{}
\setlength{\parskip}{\baselineskip}%
\setlength{\parindent}{0pt}
\begin{document}
\maketitle
\section{}
\section{}
\textbf{Lemma A:}

\textbf{Statement:} A connected subgraph of $n$ nodes requires at least $n-1$ edges.

\textbf{Proof:} Consider a graph $G = (V,E)$. 

Let $G_n$ be the set of all connected subgraphs of $G$ with $n$ nodes. And let $G'_n$ be a subgraph of $G_n$ such that the number of edges in $G'_n$ is minimized. Then let $E'_n = E(G'_n)$ and $V'_n = V(G'_n)$.

First we shall consider the base case of $n = 1$ nodes. Here it is trivial to see by definition of a graph that $E'_1 = 0$. 

Now assume that $E'_b \geq \alpha$. Then it must be true that $E'_{b+1} \geq \alpha + 1$ since for any subgraph in $G_b$, by definition of a graph, requires at least one new edge to connect a new node.

Therefore $E'_n \geq n-1$, hence proving our statement. 
\QEDA

\textbf{Lemma B:}

\textbf{Statement:}
For a spanning tree $H$, there exists a node $v$ such that $v \in H$ and $v$ has a degree of $1$

\textbf{Proof:} Clearly no node can have a degree of $0$ or else H would not be connected and therefore not be a tree. So then it is suffice to show that if all nodes in $H$ have a degree greater than $1$, then that implies that $H$ has a cycle and therefore is not a tree. 

To show this, consider a simple traversal algorithm. In this algorithm one starts at an arbitrary node in $H$ and then traverses an unused edge until it can no longer move because it is stuck at a node with all used edges. If at any point in this algorithm it hits a node already visited then there must be a cycle in the graph since it would imply that there is a path using unique edges from a node back to itself. If this algorithm were run on $H$ such that the degree of each node in $H$ was at least $2$, then the algorithm would have to hit a node a second time. This is because for the algorithm to not find a cycle it would have to never hit a node twice and then stop at a final unvisited node because it was stuck. But it can't get stuck at such a node since the degree of all nodes is greater than $1$. Therefore there must exist a node in $H$ such that its degree is $1$.
\QEDA

\textbf{Lemma C:} 

\textbf{Statement:} If $H$ is a tree with at least $2$ nodes then there exists a node $v$ such that $v \in H$, $H \backslash v$ is a tree and $|E(H)| = |E(H \backslash \{v\})| + 1$ 

\textbf{Proof:} By Lemma B there must be a node $v \in H$ with a degree of $1$. $H$ is a tree and therefore has no cycles so $H \backslash \{v\}$ has no cycles either. And since $v$ has a degree of $1$, then there can be no edge simple path in $H$ between any two nodes in $H \backslash v$ that includes the node $v$. Then since $H$ is connected removing $v$ and the single edge in its cut would clearly retain connectivity in $H\backslash\{v\}$. Therefore since $H \backslash v$ is connected and has no cycles it is also a tree. And since $v$ has a degree of $1$, then clearly $|E(H)| = |E(H \backslash \{v\})| + 1$. 
\QEDA
 

\textbf{Lemma 2.2}

\textbf{Statement:} Let $G$ be a connected graph with $n$ nodes. Then $G$ is a spanning tree if and only if it has exactly $n-1$ nodes.


\textbf{Proof:} 
First we shall prove that if $G$ is a spanning tree then it has 1 less edges than nodes.

Let $S_n$ be the set of all subgraphs of $G$ that are spanning trees. Then let $|E_n|$ be the number of edges of used for each subgraph in $S_n$ assuming they are all the same. 

Clearly $|E_1| = 0$ by definition of a graph.

Then if $S_n = n-1$ then $S_{n+1} = n $. 
This is because Lemma C implies for all $H \in S_{n+1}$ there exists a node $v \in H$ such that $H \backslash \{v\} \in S_{n}$ and as well $|E_{n+1}| = |E_{n}| + 1$. 

Therefore by induction $G$ is a spanning tree of $n$ nodes then it contains $n-1$ edges.

Now we shall show that if $G$ is a connected graph with $n$ nodes and $n-1$ edges then it is a tree. 

If $G$ were connected and not a tree then it must have a cycle. If it had a cycle then there must be an edge $e = (u,v)$ such that there is a path $P$, without $e$, from $u$ to $v$. Clearly $G \backslash \{e\}$ is still connected since any two nodes connected via a path using $e$ could simply replace $e$ with $P$ and still have a path between them. So then if $G$ was connected and not a tree, this would imply that for some $e \in G$, $G \backslash \{e\}$ would be connected as well. But then $G$ would have $n-2$ nodes which contradicts Lemma A.

Therefore, combining both directions, a connected graph of $n$  nodes is a tree if and only if it has $n-1$ edges.
\QEDA

\section{}
\textbf{Lemma D}

\textbf{Statement:}  Let $G = (V,E)$ be a graph and let $u,w \in V$. Then if there are two different edge simple paths $A$ and $B$ from $u$ to $w$ then there is a cycle in $G$. 

\textbf{Proof:} 
Let us write the paths $A$ and $B$ as such.
\[
A = u, e_{A,1}... v_{A,i},e_{A,i}...e_{A,j}...w
\]

\[
B =  u, e_{B,1}, ...v_{B,k} e_{B,k}....e_{B,l}...w
\]
, where $i$ and $k$ are maximized under the constraints $v_{A,i} = v_{B,k}$, $i < j$, $k < l$ and $v_{A,j},e_{A,j}...w = v_{B,l},e_{B,l}...w$ such that $j=l$ and $j$ and $l$ are minimized. 

Then clearly $\{v_{A,i+1}...v_{A,j}\} \cap \{ v_{B,k+1}...v_{B,l}\} = \emptyset$ or else $i$ and $k$ would not be maximized, and 
the path $v_{A,i}...e_{A,j-1},v_{A,j}, e_{B,l-1}...,v_{B,k+1},e_{B,k}$ is a simple cycle.
\QEDA


 
\textbf{Lemma E}

\textbf{Statement:} If $H$ is a spanning tree of graph $G$ then for each pair of nodes in $H$ there exists one unique simple path between them.

\textbf{Proof:} 
First we shall show that all edge simple paths between nodes are simple paths. If there was an edge simple path between two nodes in $H$ that was not simple, then there is a portion of that path $v_i, e_i, ... e_{j-1}, v_j$ such that $v_i = v_j$. This would imply that there is a cycle in $H$ which cannot be true since $H$ is a tree. Therefore all edge simple paths in $H$ are also simple. 

For any two nodes in $H$ there is at least one path between them since $H$ is a tree. If that path is not edge simple, then it can be transformed into an edge simple path by removing all loops $e_i, ... e_{j-1}, v_j$ such that $v_i = v_j, e_i = e_{j-1}$. And as we have shown that edge simple path must be simple. Therefore there is at least one simple path between all pairs of nodes in $H$. 

If there were more than one simple path between two nodes then select two of them and by Lemma D there must exist a cycle which cannot be true since $H$ is a tree.

Therefore for any pair of nodes in $H$ there is only one edge simple path between them and it is simple.

\QEDA


\textbf{Lemma 2.7}

\textbf{Statement}: Let $H = (V,T)$ be a spanning tree of $G = (V,E)$. Let $e = vw$ be an edge in $E \backslash T$, and let $f$ be an edge of a simple path in H from $v$ to $w$. Then (a) the subgraph $H'$ obtained from adding $e$ has a unique cycle containing $e$, and (b) the subgraph $H'' = (V,T \cup \{e\} \backslash \{f\})$ is a spanning tree of $G$.

\textbf{Proof of (a):} First we can clearly see that since $H$ is a spanning tree then there is a path from $v$ to $w$ and then adding $e$ creates a cycle by definition. If adding $e$ created more than one unique cycle, that would imply that there exists at least two unique edge simple paths connecting $v$ to $w$ in $H$. This cannot be true since by Lemma D this would imply that there was a cycle in $H$ which is a tree. Therefore $e$ only introduces one unique cycle.
\QEDA 

\textbf{Proof of (b)} 

First $e$ cannot introduce a cycle because if it did that would imply that there was a simple path from $v$ to $w$ in $H \backslash \{f\}$. But this cannot be true since by Lemma E, removing $f$ breaks the one and only simple path between $v$ and $w$. 

Again by Lemma E only all pairs of nodes in $H$ connected by a simple path containing $f$ will be disconnected in $H \backslash \{f\}$. Since $v$ and $w$ are connected in $H$ using a simple path containing $f$, then all the disconnected nodes in $H \backslash \{f\}$ must be connected to either $v$ or $w$. Then all nodes in $H \backslash \{f\}$ are connected to $v$ and or $w$. Therefore since in $H \cup \{e\} \backslash \{f\}$ $v$ and $w$ are connected, then all pairs of nodes are connected. 

Therefore $H''$ is a spanning tree.
\QEDA

\section{}
\textbf{Lemma F}

\textbf{Statement:} Let $H$ be a spanning tree of some graph $G$. Then let $e = (u,v)$ be some edge in $H$. Finally let $A$ be the set of all nodes in $H$ connected by a simple path including $e$  to $u$. Then $A$ is disconnected from $H \backslash A$ in the subgraph $H \backslash \{e\}$.

\textbf{Proof:} If the above statement were not true, this would imply that the $|\delta(A)|$ of $H$ was greater than $1$. But this can't be true because then there would be more than one unique edge simple paths between pairs of nodes in $H$ which breaks lemma F.
\QEDA

\textbf{Exercise 2.9}

\textbf{Statement:} Let $G = (V,E)$ be a connected graph with costs $c_e, e \in E$. If $H = (V,T)$ is a MST of $G$, and $e \in T$ then there is a cut $D$ of $G$ with $e \in D$ and $c_e = min\{c_f : f \in D\}$.

\textbf{Proof:} Let $e = (u,v)$ be given as described in the statement. Then let $A$ be the set of nodes connected to $u$ in $H$ via a simple path using $e$. Then let $D = \delta_G(A)$. Then by Lemma F, if $a \in D$ then either $a = e$ or $a \notin T$. Therefore 
$c_e = min\{c_f : f \in D\}$ since if this were not true, then we could create a spanning tree $H \cup \{f\} \backslash \{e\} : c_f < c_e$. This would be a spanning tree with less weight than $H$ which cannot be true since $H$ is a MST.
\QEDA
 
\end{document}